\documentclass[12pt]{article}
\usepackage{blindtext}
%%\usepackage{parskip}
% \usepackage{extramarks}
\usepackage{amsmath}
\usepackage{tikz}
\usetikzlibrary{graphs,quotes,arrows.meta}
\usepackage{algorithm}
\usepackage{forloop}
\usepackage[noend]{algpseudocode}
\usepackage[english]{babel}
% \usepackage{amsthm}
% \usepackage{amsfonts}
% \usepackage{tikz}
% \usepackage[plain]{algorithm}
% \usepackage{algpseudocode}

\topmargin=-0.45in
\evensidemargin=0in
\oddsidemargin=0in
\textwidth=6.5in
\textheight=9.0in
\headsep=0.25in

\title{CSCI 5454: Algorithms: Homework 4}
\author{Ashutosh Gandhi}
\date{\today}

\begin{document}
\maketitle

\section*{Problem 1}

% Question

\subsection*{1.1 Part A} 

% Sub question
\vspace{10pt}

% \textbf{Soln:} 

% Solution.
The probability of Alice winning would be the sum of \\
- Alice gets heads in the 1st toss \\
- Alice gets tails, Bob gets heads, Alice gets heads on 3 tosses \\
- Alice gets tails, Bob gets heads, Alice gets tails, Bob gets heads, Alice gets heads on 5 tosses 
and so on. Thus we get the equation \\
\(
P = 1/2 + 1/2*1/2*1/2 + 1/2*1/2*1/2*1/2*1/2 + ..... \\
P = 1/2 + 1/2^3 + 1/2^5 + ..... --> \textcircled{1} \\
4P = 2 + 1/2 + 1/2^3 + 1/2^5 ...... --> \textcircled{2} \\
\text{subtracting \textcircled{2} - \textcircled{1} we get} \\
3P = 2 \quad \text{\# All other terms cancel out each other} \\
P = 2/3
\) \\
The probability that Alice wins is 2/3
\\~\\~\\
Let X denote a random variable representing the number of tosses until a player wins. \\
Then \(E(X)=E(X_1 + X_2 + X_3 + ...) \) \\ 
\(X_1\) represents tossing 1 coin, \(X_2\) 2 coin tosses \\ 
\(E(X)=E(X_1) + E(X_2) + E(X_3) + ...) \) \\ 
\(E(X)=1*(1/2) + 2*(1/2^2) + 3*(1/2^3) + 4*(1/2^4) + .... --> \textcircled{1} \quad E(X)=\sum xP(X=x)\) \\
\text{For one toss Alice gets head, for 2 tosses Alice gets tails and Bob gets tails,} \\ 
\text{for 3 tosses Alice gets tails, Bob gets heads, Alice gets heads} \\
\( (1/2)E(X) = 1/2^2 + 2/2^3 + 3/2^4 + .... -> \textcircled{2} \\
\text{subtracting \textcircled{1} - \textcircled{2} we get} \\
(1/2)E(X) = 1/2 + 1/2^2 + 1/2^3 + 1/2^4 + .... \\
\text{This is an infinite GP with a=1/2 and r=1/2 so sum=a/1-r} \\
(1/2)E(X) = (1/2)/(1-1/2) \\
(1/2)E(X) = 1 \\
E(X) = 2
\) 
\\
The expected number of coin tosses is 2

\subsection*{1.2 Part B} 

% Sub question
\vspace{10pt}

Consider a game G1 such that: 
\begin{itemize}
\item Alice tosses a coin, if it is head Alice wins, else bob gets a turn 
\item Bob tosses a coin, if it is tails Bob wins, else
\item Alice wins
\end{itemize} 
The probability of Bob winning in G1 is Alice getting a tail and then Bob getting a tail \(P=1/2*1/2=1/4\)
\\ \\
Consider a game G2 such that:
\begin{itemize}
    \item Toss 5 coins, If Alice gets at least five heads she wins, else Bob wins
\end{itemize}
The probability of Bob winning in G2 is if all the coin tosses flip as tails. \(P=1/32\)
\\ \\
Next, consider the composite game:
\begin{itemize}
    \item Toss 2 coins, If they are both tails then Bob wins, if they are both heads then play the game G1, else play the game G2.  
\end{itemize}
The probability of Bob winning the composite game is him either getting 2 tails or winning G1 or winning G2. \\
\(P=1/4+(1/4*1/4)+(1/2*1/32)\) \\
\(P=1/4+1/16+1/64\) \\
\(P=(16+4+1)/64\) \\
\(P=21/64\)


\subsection*{2.1 Part A} 

% Sub question
\vspace{10pt}

Considering the 3 separate cases for 1, n, and 2....n-1 \\ \\
- For 1 to be a leaf node, the element just right to 1 should be picked first. \\ Proving this by contradiction: in case, 1 was picked before 2 then 2 would definitely be the right child of 1, so 2 must be picked before 1. Next, if any other \(k>2\) is picked it would always go in the right sub-tree of 2 leaving only 1 in the left sub-tree. \\ Using the probability formula in the question with i=2 and j=1 (2 being chosen before 1 and being its ancestor)  we get the probability that 1 is a leaf node as \(1/((2-1)+1)=1/2\) \\ \\
- For n to be a leaf node, the element just left to n should be picked first.\\ Proving this by contradiction: in case, n is picked first and later n-1 is picked the n-1 would be in the left child of n, so n-1 must be picked before n. Next, if any \(k<n-1\) is picked it would go in the left sub-tree of n-1 leaving the right sub-tree with only n.\\ Using the probability formula in the question with i=n-1 and j=n then we get the probability that n is a leaf node is \(1/(n-(n-1)+1)=1/2\) \\ \\
- For any element k in [2, n-1], k can only be a leaf node if it is picked after both its left and right neighbors are picked.\\ Proving this by contradiction: in case, k is picked before k-1 then k-1 and any other element less than k-1 till 1 can go in the left sub-tree of k, moreover, any remaining element k+1 to n would go in its right sub-tree. Similarly, if k is picked before k+1 then k+1 and any element more than k+1 till n can go to the right sub-tree of k, moreover, any remaining element k-1 to 1 would go into its left sub-tree. So for k to be a leaf node it has to be picked only after k+1 and k-1 are picked.\\ So out of k-1, k, k+1 we have 6 possible ways of picking them, out of which only 2 have k being selected after k-1 and k+1. So the probability would be 2/6 or 1/3.  

\subsection*{2.2 Part B} 

% Sub question
\vspace{10pt}
Let X denote an indicator variable which takes the value 1 if node i is a leaf node or 0 if i is not a leaf node \\
number of leaf nodes = \(\sum_{i=1}^n[\text{i is a leaf node}]\) \\ 
Excepted number of leaf nodes E(X) = \(\sum_{i=1}^nP(\text{i is a leaf node})\) \\
\(
E(X)=1/2+ \sum_2^{n-1}1/3 + 1/2 \quad \text{(from 2a P(x=1,n)=1/2 and P(x=2...n-1)=1/3)} \\
E(X)=1+((n-2)/3) \\
E(X)=(n+1)/3 \\
\)

Thus the expected number of leaf nodes is (n+1)/3

\subsection*{3.1 Part A} 

% Sub question
\vspace{10pt}

To have \(Hx=Hy\) we can say \(Hx\oplus Hy =0\) based on the \(\oplus\) truth table. \\
or \(H.(x\oplus y)=0\) \\
Since it is given that x and y only differ in \(i^{th}\) bit, \(x \oplus y\) would be 0 for all the bits in x and y except the \(i^{th}\) bit which would be 1. \\
The n bit \(x \oplus y\) would be multiplied with  the m*n H matrix. So the \(i^{th}\) bit in \(x \oplus y\) is multiplied with the \(i^{th}\) column in H. So if we have the \(i^{th}\) column in H as all 0, then the Boolean matrix multiplication would also be all 0, since \(0 \& 1=0\), \(0 \& 0=0\)  and \(0 \oplus 0=0\)
\\ \\
Thus, the condition to have \(Hx=Hy\) with the \(i^{th}\) bit differing in x and y, is to have the \(i^{th}\) column in H be all zeros. 

\subsection*{3.2 Part B} 

% Sub question
\vspace{10pt}

To have the 2 keys x and y collide for a random matrix H we need to have \(Hx=Hy\) or \(H.(x\oplus y)=0\) \\

\(x\oplus y\) would be a vector of zeros except at the bits where x and y are different. Now to get \(H.(x\oplus y)=0\) the XOR of those columns(the bits where x and y differ) in H must be 0. For example, let \(V_6, V_8, V_{13}, V_{19}, V_{45}\) be some 5 columns in the matrix H and 6,8,13,19 and 45 be the bits at which x and y differ, then \(V_6 \oplus V_8 \oplus V_{13} \oplus V_{19} \oplus V_{45}\) is a m bit vector of zeros. \(V_6, V_8, V_{13}, V_{19}, V_{45}\) is m bit vector/columns in H and let b denote the bit in the 1st row of each of these vectors, then the probability of \(b_6 \oplus b_8 \oplus b_{13} \oplus b_{19} \oplus b_{45}=0\) being would be 1/2, in fact, for any k such additions it would 1/2. 
\\ \\
To prove \(P(b_1 \oplus b_2 \oplus b_3 \oplus ..... \oplus b_k=0)=1/2\). \\
If we fix k-1 bits then they would all, either XOR to a 0 or 1, and the \(k^{th}\) bit would also be either 0 or 1. So both of them would match at 2 out of the 4 possible outcomes and at those 2 matches the XOR would be 0. Thus the probability of getting the above XOR sum to 0 is 1/2. 
\\ \\
Since The matrix H is of dimension m*n we would be doing the above XOR operation for each of the m rows, and the probability for each row having an XOR sum of 0 is 1/2. So the overall probability of having \(H.(x\oplus y)=0\) is \((1/2)^m\). \\ \\
Thus to have a collision of keys x and y the condition \(H.(x\oplus y)=0\) must be true and the probability of that happening is \(1/2^m\) as proved above.
\end{document}
