\documentclass[12pt]{article}
\usepackage{blindtext}
%%\usepackage{parskip}
% \usepackage{extramarks}
\usepackage{amsmath}
\usepackage{tikz}
\usetikzlibrary{graphs,quotes,arrows.meta}
\usepackage{algorithm}
\usepackage{hyperref}
\usepackage{forloop}
\usepackage[noend]{algpseudocode}
\usepackage[english]{babel}
% \usepackage{amsthm}
% \usepackage{amsfonts}
% \usepackage{tikz}
% \usepackage[plain]{algorithm}
% \usepackage{algpseudocode}

\topmargin=-0.45in
\evensidemargin=0in
\oddsidemargin=0in
\textwidth=6.5in
\textheight=9.0in
\headsep=0.25in

\title{CSCI 5454: Algorithms: Homework 5}
\author{Ashutosh Gandhi}
\date{\today}

\begin{document}
\maketitle

\section*{Problem 1}

% Question

\subsection*{1.1 Part A} 
% Sub question
\vspace{10pt}
% \textbf{Soln:} 
% Solution.
P(a ball goes to bin j) = \(1/m\) \\
P(a ball does not go to bin j) = \(1-1/m\) \\
P(none of the nk balls go to bin j) = \((1-1/m)^{nk}\)

\subsection*{1.2 Part B} 
% Sub question
\vspace{10pt}
P(one bin is unoccupied) = \((1-1/m)^{nk}\) \# from 1.a \\
P(one bin is occupied) = \(1-(1-1/m)^{nk}\) \\
P(k bins are occupied) = \((1-(1-1/m)^{nk})^k\) \\
using the inequality \(1-x &\leq e^{-x}\) \\ 
\begin{equation}
\begin{aligned}
\nonumber
(1-1/m) &\leq e^{-1/m} \\
(1-1/m)^{nk} &\leq e^{-nk/m} \\
-(1-1/m)^{nk} &\geq -e^{-nk/m} \\
1-(1-1/m)^{nk} &\geq 1-e^{-nk/m} \\
(1-(1-1/m)^{nk}))^k &\geq (1-e^{-nk/m})^k \quad \text{\# since $k > 0$ }\\
\end{aligned}
\end{equation}
\\
Thus, P(k bins are occupied) \( \geq (1-e^{-nk/m})^k \)

\subsection*{1.2 Part C} 
% Sub question
\vspace{10pt}

Let \(E_1\) denote probability that bin 1 is occupied, \(E_2\) that bin 2 is occupied, \(E_1 \cap E_2\) denote that bin1 and bin2 are both occupied, and \(E_1 \cup E_2\) denote that either bin1 or bin2 is both occupied \\
Then, \(P(E_1 \cap E_2)=P(E_1)+P(E_2)+P(E_1 \cup E_2))\) \\
from 1.a; P(none of the q balls goes to a particular bin) = \((1-1/m)^{q}\) \\
so P(\(E_1)\)) = \(1-(1-1/m)^{q}\) \\
similarly P(\(E_2)\)) = \(1-(1-1/m)^{q}\) \\
\\
For finding \(E_1 \cup E_2\) \\
P(a ball not going to bin1 or bin2) = (m-2)/m \# since it could go in any of the remaining m-2 bins \\ 
P(neither bin1 or bin2 is occupied after q balls) = \((1-2/m)^q\) \\
P(bin1 or bin 2 is occupied) = \(1-(1-2/m)^q\) \\
\(P(E_1 \cup E_2) = 1-(1-2/m)^q\) \\~\\
Thus, \(P(E_1 \cap E_2)=2[1-(1-1/m)^{q}] - [1-(1-2/m)^q] \) \\ 
\(P(E_1 \cap E_2)= 1 - 2(1-\frac{1}{m})^q + (1-\frac{2}{m})^q\) \\

\subsection*{1.2 Part D} 
% Sub question
\vspace{10pt}

from 1.2 False positive rate \( \approx  (1-e^{-nk/m})^k\) \\
so \(0.01 \approx (1-e^{-nk/m})^k\), with m=\(8*10^6, k=10\) \\
Computing the above equation on WolframAlpha \(^{[1]}\) we get n \(\approx\) 797474 \\
Thus, the number of unique elements that can be inserted is 797,474 \\~\\
In the case of a hashmap, the total storage would be 40*n bytes = \(797474*40)\) Bytes or 30.42 MB \\
The bloom filter takes almost 1MB size, thus the overall size save is \(30.42-0.95 \approx 29.47 MB\)
\\
\([1]\) \url{https://www.wolframalpha.com/input?i=%281-e%5E%28%28-n%29*%2810%2F8000000%29%29%29%5E10%3D0.01}
\subsection*{2.1 Part A} 

% Sub question
\vspace{10pt}

If a graph has a min-cut of k, then it must have at least nk/2 edges \\ 
P(choosing a min-cut in 1st step of contraction) = \(k/(nk/2)=2/n\) \\ 
P(not choosing a min-cut in 1st step of contraction)= \(1-2/n=(n-2)/n\) \\
Similarly, P(not choosing a min-cut in 2nd step of contraction)= \(1-2/n=(n-3)/(n-1)\) \# from 1-(k/((n-1)*k/2))\\
P(preserve a min-cut from \(G_n\) to \(G_{n/4}\)) \\ 
\begin{align*}
 &= \frac{n-2}{n} * \frac{n-3}{n-1} * \frac{n-4}{n-2} * \frac{n-5}{n-3} * ..... * \frac{n/4}{n/4+2} * \frac{n/4-1}{n/4+1}  \\
 &= \frac{n/4*(n/4-1)}{n*(n-1)} \\
 &= \frac{n-4}{16(n-1)} \\
 &\approx 1/16 \quad \text{as n tends to infinity}
\end{align*}

\subsection*{2.2 Part B} 

% Sub question
\vspace{10pt}
The time taken to contract an edge is O(n) and to do the same for n/4 edges is \(O(n^2/4)\) or \(O(n^2)\) \\
Thus the overall time complexity is given by the recurrence \\ 
\(T(n) = O(n^2) + 4T(n/4)\) \\ 
4T(n/4) since there are 4 recursive calls being made for the n/4 contracted edge graph  \\ 
Using master theorem with a=4, b=4 and c=2 we get \(\log_{b}a < c\) Thus we use case 3 of the theorem and get T(n) = \(O(n^c)\) \\~\\
Thus, \(T(n)=O(n^2)\) 

\subsection*{2.2 Part C} 

% Sub question
\vspace{10pt}
Let \(P_n\) be the probability of the Recursive Algorithm succeeding then \\ 
\(P_n\) = P(1st contraction succeeds)*P(The 4 recursive calls succeeding) \\
\(P_n\) = P(1st contraction succeeds)*[1-\((1-P_{n/4})^4\)] \\
\begin{align*}
    P_n &= 1/16*[1-(1-P_{n/4})^4] \quad \text{\# let \(4^k\) = n} \\ 
    p(k) &= 1/16*[1-(1-p(k-1))^4] \quad \text{\# let d(k) = 1/p(k)} \\
    d(k) &= \frac{16}{1-[1-1/d(k-1)]^4} \\
    d(k) &= \frac{16d(k-1)^4}{d(k-1)^4-(d(k-1)-1)^4} \quad \text{\# let d(k-1) = x} \\
    d(k) &= \frac{16x^4}{x^4-(x-1)^4} \\
    &= \frac{16x^4}{4x^3-6x^2+4x-1} \\
    d(k) &= 4x + \frac{24x^3-16x^2+4x}{4x^3-6x^2+4x-1}
 \end{align*}

 \begin{figure}[h]
    \centering
    \includegraphics[width=0.8\textwidth]{2c.png}
    \caption{Graph of $\frac{24x^3-16x^2+4x}{4x^3-6x^2+4x-1}$}
    \label{fig:my_image}
\end{figure}

As can be seen from Figure 1 that the graph is a decreasing function with its upper bound at x=1, since x is the inverse of probability it must be greater than or equal to 1. At x=1 the polynomial is equal to 12. 

\begin{align*}
    d(k) &\leq 4x + 12 \quad \text{substitute back x} \\
    d(k) &\leq 4d(k-1) + 12 \\
    &\leq 4^2d(k-2) + [12 + 4*12] \\ 
    &\leq 4^3d(k-3) + [12 + 4*12 + 4^2*12] \quad \text{going till k-1 to get the base base d(1)=1} \\
    &\leq 4^{k-1}d(1) + [12 + 4*12 + 4^2*12 + .... + 4^{k-1}*12] \\ 
    &\leq 4^{k-1} + 12[\frac{4^{k-1}-1}{4-1}] \quad \text{apply GP sum formula with a=1, r=4, n=k-1} \\
    &\leq 4^{k-1} + 4*(4^{k-1}-1) \\
    &\leq 5*4^{k-1} - 4 \quad \text{substitute back p(x)} \\
    p(k) &\geq \frac{1}{5*4^{k-1} - 4} \quad \text{substitute back n} \\
    p(n) &\geq \frac{1}{5*4^{\log_{4}n-1}-4} \\
    &\geq \frac{1}{(5/4)*n-4} \\
    p(n) &\geq \frac{4}{5n-16} \\
\end{align*}

Thus we get the lower bound \(P(n)=\Omega( 1/n)\)
\\~\\~\\
\subsection*{3.1 Part A} 

% Sub question
\vspace{10pt}
The time taken to contract an edge is O(n) and to do the same for n/2 edges is \(O(n^2/2)\) or \(O(n^2)\) \\
Thus the overall time complexity is given by the recurrence \\ 
\(T(n) = O(n^2) + 2T(n/2)\) \\ 
2T(n/2) since there are 2 recursive calls being made for the n/2 contracted edge graph \\ 
Using master theorem with a=2, b=2 and c=2 we get \(\log_{b}a < c\) Thus we use case 3 of the theorem and get T(n) = \(O(n^c)\) \\~\\
Thus, \(T(n)=O(n^2)\) 

\subsection*{3.2 Part B} 

% Sub question
\vspace{10pt}
Similar 2.a \\
P(preserve a min-cut from \(G_n\) to \(G_{n/2}\)) \\ 
\begin{align*}
 &= \frac{n-2}{n} * \frac{n-3}{n-1} * \frac{n-4}{n-2} * \frac{n-5}{n-3} * ..... * \frac{n/2}{n/2+2} * \frac{n/2-1}{n/2+1}  \\
 &= \frac{n/2*(n/2-1)}{n*(n-1)} \\
 &= \frac{n-2}{4(n-1)} \\
 &\approx 1/4 \quad \text{as n tends to infinity}
\end{align*}

Let \(P_n\) be the probability of the Recursive Algorithm succeeding then \\ 
\(P_n\) = P(1st contraction succeeds)*P(The 2 recursive calls succeeding) \\
\(P_n\) = P(1st contraction succeeds)*[1-\((1-P_{n/2})^2\)] 

\begin{align*}
    P_n &= 1/4*[1-(1-P_{n/2})^2] \quad \text{\# let \(2^k\) = n} \\ 
    p(k) &= 1/4*[1-(1-p(k-1))^2] \quad \text{\# let d(k) = 1/p(k)} \\
    d(k) &= \frac{4}{1-[1-1/d(k-1)]^2} \\
    d(k) &= \frac{4d(k-1)^2}{d(k-1)^2-(d(k-1)-1)^2} \quad \text{\# let d(k-1) = x} \\
    d(k) &= \frac{4x^2}{x^2-(x-1)^2} \\
    &= \frac{4x^2}{2x-1} \\
    d(k) &= 2x + \frac{2x}{2x-1}
 \end{align*}

\begin{figure}[h]
    \centering
    \includegraphics[width=0.8\textwidth]{2c.png}
    \caption{Graph of $\frac{2x}{2x-1}$}
    \label{fig:my_image}
\end{figure}

As can be seen from Figure 2 that the graph is a decreasing function with its upper bound at x=1, since x is the inverse of probability it must be greater than or equal to 1. At x=1 the polynomial is equal to 2. \\
Thus, 
\begin{align*}
    d(k) &\leq 2x + 2 \quad \text{substitute back d(k-1)} \\
    d(k) &\leq 2d(k-1) + 2 \\
    &\leq 2^2d(k-2) + [2 + 2*2] \\ 
    &\leq 2^3d(k-3) + [2 + 2*2 + 2^2*2] \quad \text{going till k-1 to get the base d(1)=1} \\
    &\leq 2^{k-1}d(1) + 2*[1+2*1+2^2*1+.....+2^{k-1}*1] \\
    &\leq 2^{k-1} + 2*\frac{2^{k-1}-1}{2-1} \quad \text{apply GP sum formula with a=1, r=2, n=k-1} \\
    d(k) &\leq 3*2^{k-1} - 2 \quad \text{substitute back p(x)} \\
    p(n) &\geq \frac{1}{3*2^{k-1} - 2} \quad \text{substitute back n} \\
     p(n) &\geq \frac{1}{3*2^{log_{2}2-1} -2 } \\
     p(n) &\geq \frac{1}{(3/2)*n-2} \\
     p(n) &\geq \frac{2}{3n-4} \\
\end{align*}

Thus we get the lower bound \(P(n)=\Omega( 1/n)\)

\end{document}
